
%--- This Tex file is written on Unicode UTF-8 ---%

%------------------------------------------------------------------------------------------------------------------------------------------------------------------
%------------------------------------------------------------------ PREAMBLE part (start) ---------------------------------------------------------------
%------------------------------------------------------------------------------------------------------------------------------------------------------------------
\documentclass[12pt, a4paper]{jsarticle}
\usepackage[dvipdfmx]{graphicx} % 図の挿入
\usepackage[usenames]{color} % \textcolor{red}{...} で文字や数式を色付ける
\usepackage{amsmath, amssymb} % 行列や追加の記号などのため
\usepackage{ascmac} % 囲み枠のため
\usepackage{ulem} % 下線や取り消し線のため
\usepackage{bm} % \bm{...} でベクトルを太字で書く
\usepackage{here} % [H] の指定で図をその位置に描写する
\usepackage{url} % \url{...} で URL をうまいこと書く
\usepackage[T1]{fontenc} % アンダースコアをコピペできるようにする
\allowdisplaybreaks   % <- allow new page for eqnarray
%------------------------------------------------------------------------------------------------------------
%------------------------------------------------------------------------------------------------------------
\title{Manual of the analysisHAL\_miya}
\author{
Takaya Miyamoto \\
{\small Yukawa Institute for Theoretical Physics, Kyoto University}
}
\date{\today}
%------------------------------------------------------------------------------------------------------------------------------------------------------------------
%----------------------------------------------------------------- PREAMBLE part (end) -----------------------------------------------------------------
%------------------------------------------------------------------------------------------------------------------------------------------------------------------
\begin{document}
%------------------------------------------------------------------------------------------------------------------------------------------------------------------
%--------------------------------------------------------------------- BODY part (start) --------------------------------------------------------------------
%------------------------------------------------------------------------------------------------------------------------------------------------------------------
\maketitle
\tableofcontents
\newpage
%------------------------------------------------------------------------------------------------------------
%------------------------------------------------------------------------------------------------------------
\section*{はじめに}
{\huge \textcolor{red}{このマニュアルは未完成です。\\適宜、項目を増やしていく予定です}\\}
%------------------------------------------------------------------------------------------------------------
%------------------------------------------------------------------------------------------------------------
\section{コード全体の概要}
本コードは、\textcolor{red}{HAL QCD法を用いたLattice QCD potentialの計算とその解析}を目的として構築されている。
使用する言語はC/C++もしくはpython3で、基本的な設計として以下を想定している。\\
\begin{itembox}[l]{基本設計}
	\begin{description}
		\setlength{\itemsep}{0.3cm}
		\item[C/C++] \mbox{} \\
		Single hadron correlator 及び NBS波動関数の入出力とポテンシャルの計算・出力のために用いる
		\setlength{\itemsep}{0.3cm}
		\item[python3] \mbox{} \\
		C/C++で作られたポテンシャルの解析のために用いる
	\end{description}
\end{itembox} \\

なお、C/C++では外部ライブラリ(boostなど)を使わないように設計してあるが、python3では\textcolor{red}{少なくともNumpyとScipy}、
できればMatplotlibとnpmathもインストールしていることを想定している。\\

\textcolor{blue}{また、python3コードはできるだけpython2と互換性のあるように設計しているが、確実に動くという保証はありません。}
%------------------------------------------------------------------------------------------------------------
%------------------------------------------------------------------------------------------------------------
\section{C/C++コード}
%------------------------------------------------------
\subsection{概要}
%------------------------------------------------------------------------------------------------------------
%------------------------------------------------------------------------------------------------------------
\section{python3コード}
%------------------------------------------------------
\subsection{概要}
%------------------------------------------------------------------------------------------------------------
%------------------------------------------------------------------------------------------------------------
\section{簡単な計算コード例と解析の手順}
%------------------------------------------------------
\subsection{テストデータについて}
%---------------------------
%\subsubsection{}
%------------------------------------------------------------------------------------------------------------
%------------------------------------------------------------------------------------------------------------
%\begin{thebibliography}{99}
%	\bibitem{key}
%\end{thebibliography}
%------------------------------------------------------------------------------------------------------------------------------------------------------------------
%--------------------------------------------------------------------- BODY part (end) ---------------------------------------------------------------------
%------------------------------------------------------------------------------------------------------------------------------------------------------------------
\end{document}

%\begin{itembox}[l]{Title}
%\begin{verbatim}
%
%\end{verbatim}
%\end{itembox}

%\begin{pmatrix}
%	1 & 0 \\
%	0 & 1
%\end{pmatrix}

%\mathop{\uwave{ equations }}\limits_{= 0}

%\begin{description}
%	\setlength{\itemsep}{0.3cm}
%	\item[aaa] \mbox{} \\
%	bbb
%\end{description}